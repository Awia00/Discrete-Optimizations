\section{Vehicle Routing Problems}
The \textit{vehicle routing problems} are problems which given a \textit{depot} a set of \textit{customers}, a set of \textit{routes} and a set of \textit{vehicles} tries to find a set of routes originating and terminating in the depot for each vehicle which supplies every customer. Often a minimum number of vehicles or total route length wants to be found and therefore the problems often fits the Integer Linear Programming model. 

In the \textit{Capacitated Vehicle Routing Problem} (CVRP) every customer must be visited exactly once, and each customer has a demand. Each vehicle has a capacity Q and the sum of all demands a vehicle satisfies must be less than or equal to the capacity of that vehicle. The objective is to minimize the cost of edges used in the solution. The cost of the edges could for example be gasoline spend or time taken or a combination of these. 

The \textit{Vehicle Routing Problem with Time Windows} (VRPTW) is like the CVRP in addition to the following. Each arc (directed edge) has a travel time. Each customer has a service time which is the time it takes to serve a customer. Furthermore, each customer has an associated time window within which the customer is available to be served. Multiple objectives for this problem exist but it is typically based on minimizing the number of routes.

These models are used both in Denmark and abroad. Studies have shown significant savings by using these models to plan distribution networks. Therefore it is highly relevant to come up with faster and better techniques to solve these problems.

One of the integer linear programming formulation to solve the VRPTW is the VRPTW set partitioning model. It is as follows:

\begin{alignat}{2}
min         & \sum_{p\in \Omega} c_p y_p \\
Subject to: & \sum_{p \in \Omega} a_{ip} y_p = 1 \quad  & \forall i \in V_c \\
            & y_p \in \{0,1\}                           & \forall p \in \Omega
\end{alignat}

Here $V_c$ is the set of customers, $\Omega$ is the set of all feasible CVRP routes ( note that this is factorial in the number of customers ), $c_p$ is the cost of a route $p$. $a_{ip}$ is a constant which is $1$ if route $p$ visits customer $i$ and $0$ otherwise. $y_p$ is the decision variable indicating if route $p$ is used in the solution. 

Since there is a exponential number of variables even the linear relaxation of the formulation does not seem feasibly computable for even small instances.

\section{Column Generation and Reduced Cost}
The idea behind Column Generation is to use a subset of variables(corresponding to columns in the matrix SIMPLEX model) and slowly generate more and more variables until the optimal or a good enough solution has been found. In VRPTW this technique is highly relevant since we have exponentially many variables and therefore would like to only explore some of those.

To generate these columns the concept of reduced cost is used. In SIMPLEX at each iteration the variable which leaves the non-basic variables has a positive coefficient. In any of the constraints where its coefficient is negative and a positive constant exist we know the variable can increase the current solution. Such variables are said to have \textit{negative reduced cost} (for minimization problems). The reduced cost $\bar{c}_p$ of a variable $y_p$ can be calculated as follows:

 $$\bar{c}_p = c_p - \sum_{i \in V_c} \pi_i a_{ip}$$
 
where $\pi_i$ is the dual variable corresponding to the $i^{th}$ constraint in the optimal solution. For example for a variable in the objective function $5y_1$ and part of constraints $2y_1 + ... = 4$ and $... + 1y_1 = 1$ the reduced cost is $\bar{c}_1 = 5 - 4 \pi_1 - 1 \pi_2$. 

When we generate the dual of the ILP, the constraints become the variables and vice versa. This means in practise that the dual ILP of the VRPTW set partitioning model only has a polynomial amount of variables and because we only started with a small subset of variables in the primal version, only a small set of constraints exist in the dual. Therefore we can easily solve the LP relaxation of such a dual problem and find the necessary $\pi$ variables for the reduced cost calculations.

To generate only the columns with negative reduced cost the problem of finding the minimum reduced cost can be formulated as an optimization problem: $\displaystyle \min_{p\in \Omega} \{\bar{c}_p\}$. If the solution to such a minimization problem is negative then a column with reduced cost exist otherwise the LP is optimal and no more columns can be generated, which could increase the solution. 

The LP formulation for finding the minimum reduced cost column for VRPTW is as follows:

\begin{figure}[H]
\begin{alignat}{4}
min     && \sum_{(i,j) \in A} c_{ij} x_{ij} - \sum_{i \in V_c} \pi_i a_i \\
s.t.    && x(origin(0))      & = 1 \\
        && x(ending(j))      & = x(origin(j))               && \forall j \in V_c \\
        && x(ending(n+1))    & = 1 \\
        && x(ending(j))      & \leq 1                       && \forall j \in V_c \\
        && a_i               & = x(origin(i))               && \forall i \in V_c \\
        && w_i + q_j         & \leq w_j + (1-x_{ij})M \quad && \forall (i,j) \in A \\
        && 0 \leq w_i        & \leq Q                       && \forall i \in V_c \\
        && x_{ij}            & \in \{0,1\}                  && \forall (i,j) \in A \\
        && a_{i}             & \in \{0,1\}                  && \forall i \in V_c \\
\end{alignat}
\end{figure}
Here the $x(B)$ notation is a shorthand for $\sum_{(i,j) \in B} x_{ij}$, $origin(i)$ and $ending(i)$ are shorthand for arcs respectively originating and ending in node $i$. $A$ is the set of all arcs and $V_c$ are the nodes of the customers. $q_i$ is the demands of customer $i$, $w_i$ is the amount of goods unloaded up to and including customer $i$ and $a_i$ is the decision variable indicating if customer $i$ is visited on the route. $x_{ij}$ is the decision variable indicating if arc $(i,j)$ is used in the solution, while $c_{ij}$ is the cost of the $(i,j)$-arc. Hence there is a polynomial number of constraints and variables. 

If the optimal solution to this formulation is negative then we add the route given by the $x_{ij}$ variables and resolve the LP relaxation.

\newpar A small walkthrough of the algorithm: for a depot and 3 customers we have the following demands table: In the following table we have respectively the demands of the customers, the costs of going between nodes, and the initial ILP with only three routes, each going to a customer and back again.

\begin{table}[H]
\begin{tabular}{|l|l|}
	\hline
	i & $q_i$    \\\hline
	1 & 50       \\\hline
	2 & 10       \\\hline
	3 & 25       \\\hline
\end{tabular}
\quad
\begin{tabular}{|l|llll|}
	\hline
	& $ x_0 $   & $ x_1 $ & $ x_2 $ & $ x_3 $ \\\hline
	$ x_0 $ & 0         & 25      & 75      & 30  \\
	$ x_1 $ &           & 0       & 100     & 10  \\
	$ x_2 $ &           &         & 0       & 10  \\
	$ x_3 $ &           &         &         & 0  \\
	\hline
\end{tabular}
\quad
\begin{tabular}{|lllll|}
	\hline
	& $5x_1$ & $20x_2$ & $50x_3$ &    \\\hline
	& $x_1$  &         &         & =1 \\
	&        & $x_2$   &         & =1 \\
	&        &         & $x_3$   & =1 \\\hline
\end{tabular}
\end{table}

Then we have the following costs of the routes:

\begin{itemize}
\item[] Route (0, 1, 0) costs 50
\item[] Route (0, 2, 0) costs 150
\item[] Route (0, 3, 0) costs 60
\end{itemize}
The $i^{th}$ cost will correspond to the dual variable of the $i^{th}$ constraint.

Now we pick (by guessing) the route (0,1,3,0) and calculate the reduced cost. $c_p = 50+25=75$ and $\sum_{i\in V_c} \pi_i a_{ip} = 50+60 $ which overall results in $75-110 = -35$ and hence we have a negative reduced cost for the route and we can add it to the variables of the original problem.

\newpar Note that it might be a good idea to generate multiple columns since each call to the pricing solver might be expensive. More columns should make the problem converge faster. When no columns can be found with negative reduced cost, then we have a final problem for which either the LP relaxation can be solved resulting in an upper bound for the original problem or a branch and price algorithm can be used.

\section{Branch and Price}
In the branch and price algorithm for the VRPTW first the LP relaxation is solved to optimality. Then for each arc in the graph each route will have a weight. Each of these weights are summed up. For an arc $(i,j)$ where the sum for each route using $(i,j)$ is not $1$ then we can branch on that arc either being part of a route or not. In the branch where $(i,j)$ is decided to not be in the solution then we simply removed that arc. In the opposite branch all arcs leaving $i$ and all arc entering $j$ besides $(i,j)$ are removed from the graph. This forces exactly one of the routes to enter $i$ and go to $j$ as the only option. 

Branch and price will then in general call the system all over and retry to generate more columns in each of the branch and bound nodes until the optimal integer solution is found in the linear relaxation.

\newpar An example of the branching technique for VRPTW could be as follows: A number of routes are generated and the LP relaxation generates results where the subset of routes $(1,2,3)$ have $1/3$ coefficients. The three routes go over arc $(0,1)$ and therefore the sum over that arc is $1$ but only $2$ of them traverse arc $(1,2)$ therefore the sum is $2/3$. For the sake of the example we stop calculating any more possible branching arcs and decide to branch on arc $(1,2)$. In the branch where $(1,2)$ is not in any solution the arc is removed. In the opposite branch every arc $(1, x)$, $x\neq 2$ and $(x,2)$, $x\neq 1$ are removed.
