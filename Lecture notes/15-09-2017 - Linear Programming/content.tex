
% Write a small intro here
\section{Preliminary}
A \textit{linear program}(LP) consists of a set of $n$ \textit{variables} and $m$ \textit{linear constraints}. The LP specifies an objective function which maximizes or minimizes a linear combination of the variables and their coefficients, subject to the m constraints. To formulate precisely:

% can use \nonumber\\ to remove numbers
\begin{alignat}{3}
\text{max: } &\sum_{i=1}^{n} c_i x_i\\
\text{s.t }  & \sum_{j=1}^{n} a_{ij} x_j \leq b_i && \text{ for } i=1,2,...,m\\
             & x_j \geq 0                         && \text{ for } j=1,2,...,n 
\end{alignat}

An LP can be solved with the \textit{SIMPLEX} algorithm. SIMPLEX initiates by generating the \textit{slack formulation} of the LP. To convert LP into slack form first $m$ \textit{slack variables} are introduced as follows.

\begin{alignat}{3}
\text{max: } &\sum_{i=1}^{n} c_i x_i\\
\text{s.t }  & \sum_{j=1}^{n} a_{ij} x_j + x_{n+i} \leq b_i  && \text{ for } i=1,2,...,m\\
             & x_j \geq 0                                    && \text{ for } j=1,2,...,n+m
\end{alignat}

Then the slack formulation is achieved by isolating the slack variables in the constraints and changing the maximization to an equality with an unknown constant $z$. In this formulation every variable in the $z$-row is called the \textit{non-basic} variables and the isolated variables in the constraints are called the \textit{basic} variables. The formulation is as follows:

\begin{alignat}{4}
z        &= && \sum_{i=1}^{n} c_ix_i\\
x_{n+i}  &= && b_i - \sum_{j=1}^{n} a_{ij} x_j  &&& \text{ for } i=1,2,...,m
\end{alignat}

Given the slack form SIMPLEX iteratively pivots variables to increase the objective value. In a pivot a non-basic variable essentially swaps position with a basic variable. After the pivot $z$ must either increase stay the same. The variable chosen to leave the non-basic variable is based on picking only variables with positive coefficients. The variable chosen to leave the basic variables is based on which constraint is the most binding. When it is not possible to pivot any variables without decreasing $z$, that is every coefficient of the non-basic variables are negative, the optimal solution has been found. The solution to the original problem is the set of variables, where every non-basic variable is $0$ and the basic variables are equal to the corresponding constant.

\section{Termination}
There are a finite number of basic solutions to an LP. In a basic solution there are $m$ basic variables which can be picked from the $n + m$ variables in the slack formulation. There is ${{n+m}\choose{m}} = \frac{(n+m)!}{n!*m!}$ different ways these basic variables can be picked and each basic solution has an objective value. Therefore if SIMPLEX continuously explore new basic solutions it will either terminate when all coefficients of the objective function are negative corresponding to an optimal solution or when SIMPLEX detects that the LP is \textit{unbounded}. 

\subsection{Unbounded}
In the case that the LP is unbounded the objective function will have one of its variables' coefficient positive and therefore available for pivoting. If $x_i$ has been chosen to leave the non-basic variables, but all of the coefficients of $x_i$ in the constraints are positive, then $x_i$ is effectively allowed to increase infinitely since any of the basic variables' values will also increase and therefore does not violate the non-negativity constraints. In this situation none of the constraints are therefore binding and the entire LP is unbounded hence no optimal solution exists. Here an example of an unbounded LP can be seen:

\begin{alignat}{2}
z        &= && 0 + 2 x_1 + 9x_2\\
x_3      &= && 5 + 3 x_1 - 2x_2\\
x_4      &= && 2 - 6 x_1 - 2x_2
\end{alignat}

Notice that as soon as \textit{one} of the non-basic variables are positive and it also appears as positive in just \textit{one} or more of the constraints, SIMPLEX can terminate since the LP must be unbounded.

\subsection{Degeneracy}
Until now it was assumed that SIMPLEX was able to choose a pivot which increased the objective function. For some LPs though it is possible to construct $k$ pivots from a basic solution $s_0$ where each pivot conserves the value of the objective function. The main concern appears when the $k^{th}$ pivot results in a basic solution $s_k = s_0$. This concept is called cycling and happens when specific rules are used to decide the pivoting. For example if the entering variable is decided based on picking the non-basic variable with the largest positive coefficient and the ties for most binding constraint are broken based on smallest subscript, cycles can occur. 

%Since LP has a finite number of states we can conclude that if SIMPLEX does not terminate then it must be cycling, otherwise it would imply that there is an infinite number of basic solutions.

To avoid cycling two different methods for picking the pivoting variables are presented. The perturbation method which uses an ordered set of $m$ significantly decreasing epsilons with values from $1$ to $0$. That is 

\begin{alignat}{1}
1 >> e_1 >> e_2 >> ... >> e_{m-1} >> e_m > 0
\end{alignat}

Each of these epsilons are then added to the constraints to make sure that it is impossible to have two basic solutions with the same objective value, hence eliminating degeneracy and in turn cycles. Therefore the slack formulation is now as follows:

\begin{alignat}{4}
z        &= && \sum_{i=1}^{n} c_ix_i\\
x_{n+i}  &= && e_i+b_i - \sum_{j=1}^{n} a_{ij} x_j  &&& \text{ for } i=1,2,...,m
\end{alignat}

In practise the epsilons are not materialized but are based on symbols to circumvent the otherwise expensive calculation of all the epsilons.

The second method for avoiding cycling is to use small indices as the method for chosen the pivot. That is for any two non-basic variables $x_i$ and $x_j$, $i < j$ with positive coefficients, $x_i$ is picked as the entering variable. To leave the basic variables the most binding constraint is still picked but in the case of ties, the smallest indices rule applies again. This simple rule has been proved to prevent cycles in SIMPLEX by R.G. Bland\footnote{Hence its other name "Bland's rule"} in 1977 and is used more often in practise than the perturbation method.

\section{Auxiliary LP}
Sometimes the first basic solution is infeasible, such a situation occurs when a constraint requires a variable to be greater than zero, that is one of the constraints are in the following form $x_i \geq b $ for  $ b > 0 $. This makes the first basic solution where all the non-basic variables are $0$ infeasible. 

To find the first feasible solution the \textit{auxiliary} LP form($L_{aux}$) is introduced. $L_{aux}$ is both feasible and bounded and the optimal value of $L_{aux}$ will both indicate whether the original LP is feasible and in that case its first feasible basic solution. For an LP in the following standard form:

\begin{alignat}{3}
\text{max: } &\sum_{i=1}^{n} c_i x_i\\
\text{s.t }  & \sum_{j=1}^{n} a_{ij} x_j \leq b_i  && \text{ for } i=1,2,...,m\\
             & x_j \geq 0                                    && \text{ for } j=1,2,...,n
\end{alignat}

The auxiliary LP has the following form:

\begin{alignat}{3}
\text{max: } & -x_0\\
\text{s.t }  & \sum_{j=1}^{n} a_{ij} x_j -x_0 \leq b_i  && \text{ for } i=1,2,...,m\\
             & x_j \geq 0                                    && \text{ for } j=0,1,2,...,n+m
\end{alignat}

As shown a new variable $x_0$ is introduced both in the objective function where its negation must be maximized but also in each constraint as a subtraction. The new LP is solved by SIMPLEX until an optimal solution is found.

\newpar There are two possible results to the $L_{aux}$. Either $x_0 = 0$ or $x_0 > 0$. First it must be shown that if the original LP($L$) is feasible then the optimal objective value of $L_{aux}$ is $0$. For a feasible solution $S$ of $L$ let $s_0 = 0$ and add it to $S$. $S$ must now be a feasible solution to $L_{aux}$ with objective value $0$\footnote{Since $x_0$ is the only appearing variable of the $L_{aux}$ in the objective function}. Since $x_0 \geq 0$, $x_0 = 0$ is the optimal value for maximizing $-x_0$ and therefore also optimal for entire $L_{aux}$. For an optimal solution $S$ to $L_{aux}$ with $x_0 = 0$, $x_0$ does not influence $L$ in any way and therefore $S$ is still an optimal solution for $L$. 

In the other case where $x_0 > 0$, $L$ must be infeasible. Assume that $L$ is infeasible, the optimal objective value of $L_{aux}$ cannot be $0$ and since $x_0$ therefore is positive and is part of the feasible solution $(s0,0,0,...,0)$, the objective value must be negative.

