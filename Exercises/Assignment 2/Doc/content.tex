% !TeX spellcheck=en_GB
\section{Theoretical part}
\begin{math}
Pr[C'\text{ has } \frac{n}{2} \text{ elements and has cost } \leq \mathcal{O}\left(OPT\right)]\\
\\
Pr[C' \text{ does not cover at least } \frac{n}{2} \text{ elements}] \leq \frac{n}{2} \frac{1}{4n} \leq \frac{1}{8}\\
\\
Pr[C' \text{ does cover at least } \frac{n}{2} \text{ elements}] \geq \frac{7}{8}
\end{math}

\section{Implementation part}
The computation table for the algorithms ca be seen below. The $f$ values are $scpa3: 17$, $scpc3: 18$, $scpnrf1: 132$ and $scpnrg5: 41$.

\begin{table}[H]
	\begin{tabularx}{\linewidth}{|X|l|r|X|X|}
		\hline
		\textbf{Method} & \textbf{Instance} & \textbf{Best value} & \textbf{Computation time (ms)} & \textbf{Ratio to optimal}\\\hline
		CPLEX & scpa3 & 232 & 1.141 & 1\\\hline
		CPLEX & scpc3 & 243 & 4.240 & 1\\\hline
		CPLEX & scpnrf1 & 14 & 44.488 & 1\\\hline
		CPLEX & scpnrg5 & DNF* & > 5.400.000 & N/A\\\hline\hline
		Simple Rounding & scpa3 & 449 & 52 & 1.94\\\hline
		Simple Rounding & scpc3 & 544 & 101 & 2.24\\\hline
		Simple Rounding & scpnrf1 & 76 & 728 & 5.43\\\hline
		Simple Rounding & scpnrg5 & 585 & 1.692 & 3.48**\\\hline\hline
		Randomized Rounding & scpa3 & 449 & 59 & 1.94\\\hline
		Randomized Rounding & scpc3 & 502 & 164 & 2.07\\\hline
		Randomized Rounding & scpnrf1 & 35 & 628 & 2.5\\\hline
		Randomized Rounding & scpnrg5 & 463 & 1.013 & 2.76**\\\hline\hline
		Primal Dual & scpa3 & 448 & 92 & 1.93\\\hline
		Primal Dual & scpc3 & 463 & 52 & 1.91\\\hline
		Primal Dual & scpnrf1 & 43 & 175 & 3.07\\\hline
		Primal Dual & scpnrg5 & 412 & 201 & 2.45**\\\hline
	\end{tabularx}
	\caption{Contains results for each method applied on each instance.\\ *: The best found solution after 90 minutes has value 168.\\ **: 168 is used as optimal value in these calculations.}
\end{table}

All of the results provide solutions which are closer to optimal the guaranteed $f-OPT$. For all the algorithms, the solutions to instance $scpnrf1$ is further away than any of the other instances. This corresponds nicely to the fact that $f$ value of the instance is much higher than the others. Generally there is a trend between ratio to optimality and the $f$ values, which while nice might be purely coincidental.

\noindent
The implemented approximation algorithms can be seen in \textit{SimpleRoundingApproximator.java}, \textit{RandomizedRoundingApproximator.java} and \textit{PrimalDualSchemaApproximator.java}. As the name implies we decided to solve 2.3 with the Primal Dual Schema algorithm presented in the text by Vazirani.
