% !TeX spellcheck=en_GB
\section{Theoretical part}
Unfortunately, we have not had any lectures on probabilities of random variables, and therefore, we do not have the know-how to give this proof.

\newpar If we gave a solution, it would seem that we know how to solve this kind of exercise, but it would merely be a coincidence, if it was correct.

\section{Implementation part}
\begin{table}[H]
	\centering
	\begin{tabularx}{\linewidth}{|X|l|r|X|X|}
		\hline
		\textbf{Method} & \textbf{Instance} & \textbf{Best value} & \textbf{Computation time (ms)} & \textbf{Ratio to optimal}\\\hline
		CPLEX & scpa3 & 232 & 1.141 & 1\\\hline
		CPLEX & scpc3 & 243 & 4.240 & 1\\\hline
		CPLEX & scpnrf1 & 14 & 44.488 & 1\\\hline
		CPLEX & scpnrg5 & DNF* & $>$ 5.400.000 & N/A\\\hline\hline
		Simple Rounding & scpa3 & 449 & 52 & 1.94\\\hline
		Simple Rounding & scpc3 & 544 & 101 & 2.24\\\hline
		Simple Rounding & scpnrf1 & 76 & 728 & 5.43\\\hline
		Simple Rounding & scpnrg5 & 585 & 1.692 & 3.48**\\\hline\hline
		Randomized Rounding & scpa3 & 449 & 59 & 1.94\\\hline
		Randomized Rounding & scpc3 & 502 & 164 & 2.07\\\hline
		Randomized Rounding & scpnrf1 & 35 & 628 & 2.5\\\hline
		Randomized Rounding & scpnrg5 & 463 & 1.013 & 2.76**\\\hline\hline
		Primal Dual & scpa3 & 448 & 92 & 1.93\\\hline
		Primal Dual & scpc3 & 463 & 52 & 1.91\\\hline
		Primal Dual & scpnrf1 & 43 & 175 & 3.07\\\hline
		Primal Dual & scpnrg5 & 412 & 201 & 2.45**\\\hline
	\end{tabularx}
	\caption{Results for each method applied on each instance.\\ $*$: The best found solution after 90 minutes has value 168.\\ $**$: 168 is used as optimal value in these calculations.}
\end{table}



\noindent
The implemented approximation algorithms can be seen in \textit{SimpleRoundingApproximator.java}, \textit{RandomizedRoundingApproximator.java} and \textit{PrimalDualSchemaApproximator.java}. As the name implies we decided to solve 2.3 with the Primal Dual Schema presented in the text by Vazirani.
